% Created 2016-08-05 Fri 09:35
\documentclass[bigger]{beamer}
\usepackage[utf8]{inputenc}
\usepackage[T1]{fontenc}
\usepackage{fixltx2e}
\usepackage{graphicx}
\usepackage{longtable}
\usepackage{float}
\usepackage{wrapfig}
\usepackage{rotating}
\usepackage[normalem]{ulem}
\usepackage{amsmath}
\usepackage{textcomp}
\usepackage{marvosym}
\usepackage{wasysym}
\usepackage{amssymb}
\usepackage{hyperref}
\tolerance=1000
\usepackage{xeCJK}
\setCJKmainfont[BoldFont=STZhongsong, ItalicFont=STKaiti]{STSong}
\setCJKsansfont[BoldFont=STHeiti]{STXihei}
\setCJKmonofont{STFangsong}
\usetheme{Madrid}
\author{crackhopper}
\date{2016-07-27 14:28:31}
\title{Data Exploration}
\hypersetup{
  pdfkeywords={},
  pdfsubject={data exploration},
  pdfcreator={Emacs 24.5.1 (Org mode 8.2.10)}}
\begin{document}

\maketitle
\begin{frame}{Outline}
\tableofcontents
\end{frame}


\section{Some Found Error in Data}
\label{sec-1}
\begin{frame}[label=sec-1-1]{Some Little Error in Data}
The feature map file is uncorrect. "参数对照表.csv"
\begin{figure}[htb]
\centering
\includegraphics[width=11cm]{./images/16.0727.170448.png}
\end{figure}

We have feature "FRB4" here
\begin{figure}[htb]
\centering
\includegraphics[width=7cm]{./images/16.0727.170656.png}
\end{figure}
\end{frame}
\begin{frame}[fragile,label=sec-1-2]{Some Little Error in Data}
 But, we lose "FRB4" in our data, (in original data we also have feature
"T")

\begin{verbatim}
feats.Properties.VariableNames
\end{verbatim}
\includegraphics[width=.9\linewidth]{./images/16.0727.170753.png}
\end{frame}

\section{Explore the Labels}
\label{sec-2}
\begin{frame}[fragile,label=sec-2-1]{label F}
 code:
\begin{verbatim}
hist(labs.F(abs(labs.F)<6000))
\end{verbatim}

\begin{figure}[htb]
\centering
\includegraphics[width=7cm]{./images/16.0727.154614.png}
\end{figure}
\end{frame}
\begin{frame}[fragile,label=sec-2-2]{label T}
 code:
\begin{verbatim}
hist(labs.F(abs(labs.F)<100))
\end{verbatim}

\begin{figure}[htb]
\centering
\includegraphics[width=7cm]{./images/16.0727.155045.png}
\end{figure}
\end{frame}
\begin{frame}[fragile,label=sec-2-3]{label V}
 code:
\begin{verbatim}
hist(labs.V(abs(labs.V)<1))
\end{verbatim}

\begin{figure}[htb]
\centering
\includegraphics[width=7cm]{./images/16.0727.155213.png}
\end{figure}
\end{frame}
\begin{frame}[fragile,label=sec-2-4]{Conclusion}
 \begin{block}{Conclusion}
we should seperate data when we predict F and T.
\end{block}
\begin{block}{Codes}
\begin{verbatim}
featF = feats(~Fzero,:);
labF = labs(~Fzero,:);
featT = feats(~Tzero,:);
labT = labs(~Tzero,:);
featV = feats;
labV = labs;
\end{verbatim}
\end{block}
\end{frame}
\section{Explore the Features}
\label{sec-3}
\begin{frame}[fragile,label=sec-3-1]{Feature Category}
 \begin{block}{Feature Category}
\begin{itemize}
\item machine parameters
\item geometrical parameters
\end{itemize}
\end{block}
\begin{block}{Codes}
\begin{verbatim}
featureNames = featF.Properties.VariableNames;
machine_para_idx = ...
  arrayfun(@(n) isempty(n{1}), ...
           regexp(featureNames,'\d'));
geometry_para_idx = ~machine_para_idx;
\end{verbatim}
\end{block}
\end{frame}
\begin{frame}[label=sec-3-2]{Feature Zero Count}
NOTE:
\begin{description}
\item[{1-50}] machine parameters
\item[{51-81}] geometrical parameters
\end{description}
\begin{figure}[htb]
\centering
\includegraphics[width=7cm]{./images/16.0727.160122.png}
\end{figure}
\end{frame}
\begin{frame}[fragile,label=sec-3-3]{Machanical Feature of a Large Zero Count}
 \begin{block}{Code}
\begingroup 
\footnotesize
\begin{verbatim}
sel = (featureZeroCount > 10000) & machine_para_idx;
featuresManyZero = featF(:,sel);
for token = featuresManyZero.Properties.VariableNames
  t = token{:};
  fmat = table2array(featuresManyZero(:,t));
  fprintf('%s, %s, %d\n',t ...
         ,getFeatMeaning(t,feat_meaning,feat_token) ...
         ,sum(fmat==0));
end
\end{verbatim}
\endgroup
\end{block}
\begin{block}{Result}
\begingroup 
\footnotesize
\begin{verbatim}
PDWDL, 盾尾密封前部左下压力, 18341
PZUL, 左上注浆压力, 45819
PZUR, 右上注浆压力, 45730
PZDR, 右下注浆压力, 65981
PZDL, 左下注浆压力, 51743
\end{verbatim}
\endgroup
\end{block}
\end{frame}
\begin{frame}[label=sec-3-4]{Machanical Feature of a Large Zero Count}
\begin{figure}[htb]
\centering
\includegraphics[width=9cm]{./images/16.0727.172847.png}
\end{figure}
\end{frame}
\begin{frame}[label=sec-3-5]{Machanical Feature of a Large Zero Count}
How to process these features?

\vspace{10pt}

for : PZUL,PZUR,PZDL,PZDR

\vspace{10pt}

\begin{itemize}
\item Ignore them [we just ignore them now]
\item Seperate the zero and non-zero, then process the data
\item Drop the features
\item Turn into Discrete features
\end{itemize}
\end{frame}
\begin{frame}[fragile,label=sec-3-6]{Feature Quantiles}
 \begin{verbatim}
ds=table2dataset(featF);
a=summary(ds);
b=[a.Variables(:).Data];
c=[b(:).Quantiles];
\end{verbatim}


\begin{figure}[htb]
\centering
\includegraphics[width=10cm]{./images/16.0727.180851.png}
\end{figure}

\begin{figure}[htb]
\centering
\includegraphics[width=10cm]{./images/16.0727.181141.png}
\end{figure}
\end{frame}
\begin{frame}[label=sec-3-7]{Feature Quantiles}
\begin{block}{Problems}
\begin{itemize}
\item Many feature concentrate around small value
\item Features have different scales
\end{itemize}
\end{block}
\begin{block}{What we can do}
\begin{itemize}
\item Outlier Detection (will force some feature to be zero)
\item Normalize all features (we do this now)
\item Equalize some features
\item Pick Out some feature to seperately handle (features of almost
Zero-value)
\end{itemize}
\end{block}
\end{frame}
\begin{frame}[label=sec-3-8]{Step Changes in Geometrical Features}
Y2,地层2比重
\begin{figure}[htb]
\centering
\includegraphics[width=7cm]{./images/16.0727.183921.png}
\end{figure}
\end{frame}
\begin{frame}[label=sec-3-9]{Step Changes in Geometrical Features}
H1,地层1厚度
\begin{figure}[htb]
\centering
\includegraphics[width=7cm]{./images/16.0727.184202.png}
\end{figure}
\end{frame}
\begin{frame}[label=sec-3-10]{Conclusions about Geometrical Features}
\begin{itemize}
\item Not very predictable
\item Can be treated as const value in local
\item Change in a jumping way
\item Some parameters change more frequently than others
\end{itemize}
\begin{block}{What can we do in future?}
\begin{itemize}
\item For less changed parameter, we can seperate dataset so that they are
not changed in a subset
\item For frequently changed parameter, maybe smoothen?
\end{itemize}
\end{block}
We don't handle this problem
\end{frame}

\section{Quick and Dirty Experiment}
\label{sec-4}
\begin{frame}[label=sec-4-1]{Data Prepare}
I drop the sample with abnormal labels, and get 3 tasks corresponding
to the prediction of each label :

\begin{description}
\item[{X$_{\text{F}}$, Y$_{\text{F}}$}] drop 0-valued F, normalize all features
\item[{X$_{\text{T}}$, Y$_{\text{T}}$}] drop 0-valued T, normalize all features
\item[{X$_{\text{V}}$, Y$_{\text{V}}$}] normalize all samples
\end{description}

Now we have 3 task, with a little different sampleset.
\end{frame}
\begin{frame}[fragile,label=sec-4-2]{Linear model Prediction of F}
 \begin{block}{Code}
\begin{verbatim}
lm=fitlm(X_tr{1},Y_tr{1});
w=lm.Coefficients.Estimate;
rmse=std(Y_te{1}-X_te{1}*w(2:nfeat+1)-w(1));
fprintf('lm :%.4f\n',rmse);
\end{verbatim}
\end{block}
\begin{block}{Result}
lm :1.5977
\end{block}
\end{frame}
\begin{frame}[label=sec-4-3]{Multitask Learning for a Trial}
\begin{block}{Multi-task Lasso with Least Squares Loss}
\begin{equation*}
\min_W \sum_{i=1}^M ||W^T_iX_i-Y_i|| + \rho_1 ||W||_1 + \rho_2 ||W||_F^2
\end{equation*}
\end{block}
\begin{figure}[htb]
\centering
\includegraphics[width=10cm]{./images/16.0727.193246.png}
\end{figure}
\end{frame}
\begin{frame}[fragile,label=sec-4-4]{Prediction of F (MTL lasso)}
 \begin{block}{Code}
\begin{verbatim}
W = Least_Lasso(X_tr, Y_tr, 0, opts);
X1=X_te{1};
Y1=Y_te{1};
W1=W(:,1);
rmse=std(Y1-X1*W1);
fprintf('mtl:%.4f\n',rmse);
\end{verbatim}
\end{block}
\begin{block}{Result}
mtl:1.9027
\end{block}
\end{frame}
\begin{frame}[label=sec-4-5]{Multitask Learning for a Trial}
\begin{block}{Multi-task with Trace-norm regularizer}
\begin{equation*}
\min_W \mathcal{L}(W) + \lambda rank(W)
\end{equation*}
\end{block}
\begin{figure}[htb]
\centering
\includegraphics[width=10cm]{./images/16.0727.215107.png}
\end{figure}
\end{frame}
\begin{frame}[fragile,label=sec-4-6]{Prediction of F (MTL trace norm)}
 \begin{block}{Code}
\begin{verbatim}
W = Least_Trace(X_tr, Y_tr, 0, opts);
X1=X_te{1};
Y1=Y_te{1};
W1=W(:,1);
rmse=std(Y1-X1*W1);
fprintf('mtl:%.4f\n',rmse);
\end{verbatim}
\end{block}
\begin{block}{Result}
mtl:49.6251
\end{block}
\end{frame}
\section{Experiment Done by Sci-kit learn}
\label{sec-5}
\begin{frame}[label=sec-5-1]{Linear Regression}
\begin{figure}[htb]
\centering
\includegraphics[width=8cm]{./images/16.0804.193322.png}
\end{figure}
\end{frame}
\begin{frame}[label=sec-5-2]{Score of Linear Regression}
\begin{equation*}
score = 1-\frac{\sum(y_{pred}-y_{true})^2}{\sum(y_{true}-y_{mean})^{2}}
\end{equation*}

\begin{figure}[htb]
\centering
\includegraphics[width=7cm]{./images/16.0804.193825.png}
\end{figure}

\begin{equation*}
rmse = \sqrt{\frac{1}{n}\sum(y_{pred}-y_{true})^2}
\end{equation*}

\begin{figure}[htb]
\centering
\includegraphics[width=9cm]{./images/16.0804.194309.png}
\end{figure}
\end{frame}
\begin{frame}[label=sec-5-3]{The Mean of F value}
The Mean of F value: 
\begin{figure}[htb]
\centering
\includegraphics[width=7cm]{./images/16.0804.194556.png}
\end{figure}
SO WE FITTED VERY WELL!!
\end{frame}
\begin{frame}[label=sec-5-4]{Highly correlated feature by Linear Regression}
We just print the highly coefficient of Linear Regression
\begin{figure}[htb]
\centering
\includegraphics[width=12cm]{./images/16.0804.201546.png}
\end{figure}
\end{frame}

\begin{frame}[label=sec-5-5]{Feature Selection : LASSO}
\begin{figure}[htb]
\centering
\includegraphics[width=11cm]{./images/16.0804.201223.png}
\end{figure}
\end{frame}

\begin{frame}[label=sec-5-6]{Feature Selection : LASSO}
the coefficient plot
\begin{figure}[htb]
\centering
\includegraphics[width=8cm]{./images/16.0804.201654.png}
\end{figure}
\end{frame}

\begin{frame}[label=sec-5-7]{Score of LASSO}
\begin{figure}[htb]
\centering
\includegraphics[width=12cm]{./images/16.0804.201835.png}
\end{figure}
\end{frame}

\begin{frame}[label=sec-5-8]{Label Distribution}
\begin{figure}[htb]
\centering
\includegraphics[width=10cm]{./images/16.0804.211703.png}
\end{figure}
\end{frame}
\begin{frame}[label=sec-5-9]{Features selected by Linear Regression}
Maybe not the right feature, but don't know why.
\begin{figure}[htb]
\centering
\includegraphics[width=8cm]{./images/16.0804.211605.png}
\end{figure}
\end{frame}

\begin{frame}[label=sec-5-10]{Features selected by LASSO}
Although we get the answer which seems more correct, but we just
predict F, by other maybe varied related machine parameters???
\begin{figure}[htb]
\centering
\includegraphics[width=10cm]{./images/16.0804.211733.png}
\end{figure}
\end{frame}

\begin{frame}[label=sec-5-11]{Correlation of Features}
\begin{figure}[htb]
\centering
\includegraphics[width=10cm]{./images/16.0804.220147.png}
\end{figure}
\end{frame}

\begin{frame}[label=sec-5-12]{PCA dimension reduction}
\begin{figure}[htb]
\centering
\includegraphics[width=10cm]{./images/16.0804.220247.png}
\end{figure}
\end{frame}

\begin{frame}[label=sec-5-13]{CCA cross decomposition}
I encountered some problem here, no graph yet.
\end{frame}
% Emacs 24.5.1 (Org mode 8.2.10)
\end{document}